%%________________________________________________________________________
%% LEIM | PROJETO
%% 2022 / 2013 / 2012
%% Modelo para relatório
%% v04: alteração ADEETC para DEETC; outros ajustes
%% v03: correção de gralhas
%% v02: inclui anexo sobre utilização sistema controlo de versões
%% v01: original
%% PTS / MAR.2022 / MAI.2013 / 23.MAI.2012 (construído)
%%________________________________________________________________________




%%________________________________________________________________________
\chapter{Implementação do Modelo}
\label{ch:implementacaoDoModelo}
%%________________________________________________________________________

O desenvolvimento da aplicação assentou em fundamentos sólidos tanto ao nível tecnológico como ao nível das boas práticas de engenharia de software e tratamento de dados. A escolha da arquitetura, das tecnologias utilizadas e dos modelos de organização dos dados foi orientada por critérios de fiabilidade, modularidade, simplicidade de manutenção e escalabilidade.

\section{Arquitetura Tecnológica}
- Falar sobre \textit{stack} usada no projeto, alternativas e prós e contras

\section{Análise dos dados}
Nesta secção iremos descrever como analisamos a informação recebida da plataforma de simulação e as decisões tomadas sobre essa informação.

\subsection{Análise inicial da informação}
Após termos recebido os dados, fizemos uma primeira análise com o objetivo de perceber a informação recebida e os próximos passos a tomar. No total, recebemos 39 ficheiros \gls{xlsx} que foram exportados da plataforma, 

\subsection{Classificação da informação}

- Falar sobre como classificamos os 39 ficheiros Excell em "buckets" e cada bucket ficou com um "template" gráfico pré-determinado
- Falar sobre cada grupo, o que significa e o que todos os gráficos partilham em comum em cada grupo.

\subsection{Visualização escolhidas para os vários grupos}
- Falar sobre os vários

\section{Processamento dos dados}
- Iniciar o tema de processamento de informação, converter informação desconhecida para um formato normalizado.

\subsection{\textit{Pipeline} de transformação dos dados}
A primeira fase de normalização segue os seguintes passos:
\begin{itemize}
    \item Extrai o nome do gráfico a partir da primeira linha de cada folha.
    \item Usa a segunda linha como cabeçalhos e normaliza os nomes (ex: substitui espaços por \textit{underscores}, remove quebras de linha).
    \item Remove colunas irrelevantes com base numa lista de columnas que não tem representação.
    \item Normaliza dados nas células (como por exemolo: remove quebras de linha, espaços duplos).
\end{itemize}

A segunda fase vai depender do ficheiro e do gráfico que queremos apresentar, mas no geral,

(...) completar isto

- Falar como foi desenvolvida, e as várias fases  de transformação A desenvolvida em \textit{Python} com recurso ao Pandas. 


\subsection{Conversão e Exportação para CSV}
- Falar sobre a transformação do pandas\n
- Falar da exportação do CSV\n
- Falar da gestão de ficheiros  CSV e \gls{xlsx}, e metodologias para não haver conflito entre ficheiros carregados\n


\section{Desenvolvimento da Aplicação Web}

Esta secção foca-se na estrutura e funcionamento da aplicação web — tanto o backend em Django como a interface construída com HTML, WebComponents e Flowbite.
\subsection{Arquitetura da aplicação Django}

    \begin{itemize}
        \item Descrever como está organizada a aplicação: os modelos principais (Quarter, ExcelFile, CSVFile), as relações entre eles e como são usados para garantir o isolamento por utilizador.
        \item Explicar o sistema de autenticação (Django Auth) e como se garantem as permissões e o acesso aos dados por utilizador.
        \item Referir também o uso do sistema de media (uploads) com UUIDs por quarter, e a criação automática das pastas.
    \end{itemize}

\subsection{Endpoints e API interna}

    Detalhar os principais endpoints utilizados: uploads, visualização de gráficos, listagem de ficheiros, etc.

    \begin{itemize}
        \item Mencionar a estrutura REST dos endpoints e como a interface os consome (por exemplo, o /api/charts/<chart_id>/<quarter>/).
        \item 
        \item Referir como se gere o estado do sistema com propriedades como \textit{is_current}, e o que acontece quando se substitui um ficheiro existente.
    \end{itemize}

\subsection{Sistema de processamento assíncrono e normalização}

    \begin{itemize}
        \item Falar sobre o \textit{run_pipeline_for_sheet}(...) do data_processing.py e como isso se encaixa com o modelo ExcelFile.
        \item 
        \item Comentar o sistema de marcação dos ficheiros como processados e o uso de flags como processed.
    \end{itemize}



    \section{Interface Gráfica e Experiência do Utilizador (Frontend)}

Aqui explicas como a interface foi pensada, as ferramentas utilizadas e como os gráficos são construídos com base nos dados enviados pelo backend.
\subsection{Design System e Flowbite}

    Explicar por que escolheste Flowbite e como ele ajudou a construir uma UI consistente e reutilizável.

    Mostrar exemplos de componentes reutilizados, como modais, botões, tabs, etc.

\subsection{WebComponents e gráficos interativos}

    Falar sobre como encapsulaste a lógica dos gráficos usando WebComponents, para evitar conflitos de JS e garantir modularidade.

    Descrever a comunicação entre os WebComponents e o backend via fetch, passando parâmetros (por exemplo, o quarter atual, ou filtros).

\subsection{Gestão de Quarters e Uploads}

    Mostrar como funciona a criação de quarters e a navegação entre diferentes trimestres (com os botões e setas).

    Explicar o fluxo de upload de ficheiros e como a interface valida o tipo de ficheiro, evita duplicações, e atualiza a visualização após carregamento.

%%________________________________________________________________________
\chapter{Validação e Testes}
\label{ch:validacaoTestes}
%%________________________________________________________________________

Validação e testes aqui \ldots; pode precisar de referir o capítulo \ref{ch:modeloProposto} ou alguma das suas secções, \eg, a secção \ref{sec:fundamentos} \ldots

Pode precisar de apresentar tabelas. Por exemplo, a tabela \ref{tab:umaTabela} apresenta os dados obtidos na experiência \ldots
\begin{table}[h]
   \centering
   \begin{tabular}{l|l|l|l}
      $c_1$ & $c_2$ & $c_3$ & $\sum_{i=1} c_i$
      \\
      \hline \hline
      $1$ & $2$ & $3$ & $6$
      \\ \hline
      $1.1$ & $2.2$ & $3.3$ & $6.6$
      \\
      \hline \hline
   \end{tabular}
\caption{Uma tabela}
\label{tab:umaTabela}
\end{table}

Para além de tabelas pode também precisar de apresentar figuras. Por exemplo, a figura \ref{fig:umafigura} descreve \ldots
\begin{figure}[h]
   \centering
   \includegraphics[width=2cm]{./fig_logo_ISEL}
\caption{Uma figura}
\label{fig:umafigura}
\end{figure}

\paragraph{Atenção.} Todas as tabelas e figuras, \eg, diagramas, imagens ilustrativas da aplicação em funcionamento, têm que ser devidamente enquadradas no texto antes de serem apresentadas e esse enquadramento inclui uma explicação da imagem apresentada e eventuais conclusões (interpretações) a tirar dessa imagem.


%%________________________________________________________________________
\chapter{Conclusões e Trabalho Futuro}
\label{ch:conclusoesTrabalhoFuturo}
%%________________________________________________________________________

Conclusões e trabalho futuro aqui \ldots

Quais as principais mensagens a transmitir ao leitor deste trabalho? O leitor está certamente interessado nos temas aqui abordados. Em geral procurará, neste projeto, pistas para algum outro objetivo. Assim, é muito importante que o leitor perceba rapidamente a relação entre este trabalho e o seu próprio (do leitor) objetivo.

Aqui é o local próprio para condensar a experiência adquirida neste projeto e apresentá-la a outros (futuros leitores).

O pressuposto é o de que de que este projeto é um \aspas{elemento vivo} que recorreu a outros elementos (\cf, capítulo \ref{ch:trabalhoRelacionado}) para ser construído e que poderá servir de suporte à construção de futuros projetos.








